\documentclass[11pt,twoside,a4paper]{article}
\usepackage{cite}
\usepackage[cm]{fullpage}
\usepackage{graphicx}
\usepackage{feynmp}
\usepackage{amsmath}
\usepackage{amssymb}
\DeclareGraphicsRule{*}{mps}{*}{}
\usepackage{caption}
\usepackage{subcaption}
\usepackage{slashed}
\usepackage[utf8]{inputenc}


\begin{document}
\begin{fmffile}{feynmandiags}

\title{18 Month ReviewResearch Plan}
\author{Patrick Dunne \\ Supervisors: David Colling, Gavin Davies}
\maketitle

My thesis will be on studies of Higgs boson physics performed at the CMS experiment at the LHC, with a focus on searches for Higgs boson decays to invisible final states, particularly those in the Vector Boson Fusion (VBF) channel, and the combination of the results of this search with those from other search channels. I began by participating, as part of the team that performed a cross-check of the main analysis, in a search for invisible Higgs boson decays in the VBF channel using data collected at the CMS detector in 2012. This search resulted in the production of a CMS approved Physics Analysis Summary (PAS) in August 2013. As the analysis progressed, in addition to cross-checking the main analysis I became responsible for the production of the final limits set on the Higgs boson invisible branching fraction. As part of this work I became the contact for the analysis in the CMS Higgs combination group, which performs the combination of results from all of the Higgs boson searches at CMS, in order to make overall measurements of Higss boson properties, and ensures that individual analyses comply with the recommendations for statistical methods to be used.

After the PAS result was completed I worked on the combination of the VBF search with two other searches performed at the CMS experiment. The other searches were performed in the channel where the Higgs boson is produced in association with a Z boson. This combination involved the refinement of the VBF analysis to ensure consistency with the other searches, and the production of new limits from the combination of the three searches, which I was responsible for. This culminated in a CMS approved result being submitted to the European Phyics Journal in spring 2014, which is currently under review and is the strongest limit on the invisible branching fraction of the Higgs boson to date.

The results discussed up until this point were obtained using 'prompt' data from the CMS experiment. In addition to this prompt data, there is also 'parked' data, which has less stringent trigger requirements than, and is recorded at the same time as, the 'prompt' data but is not reconstructed until later due to computing resource constraints. Due to an extended technical shut down of the LHC the parked data is currently the only available unanalysed data from CMS. I am now working as part of a team producing an updated VBF Higgs to invisible search using this additional parked data, with a result expected by the end of 2014. The challenge of using the parked data is that, in addition to increasing the acceptance of signal events, the looser trigger requirements also cause significantly larger numbers of background events to be selected. We hope to use multi-variate analysis techniques, such as boosted decision trees, to make the analysis of this data possible.

Depending on the outcome of this work I will move onto studying preparations for analyses after the LHC restarts in 2015. The running conditions of the LHC will be very different at this point. The centre of mass energy is being increased to 13 TeV, which will provide increased discovery potential for heavy objects, such as additional non-standard model Higgs bosons, but also there will be an increased instantaneous luminosity, leading to increased numbers of additional interactions in each event, which will present significant challenges in the areas of triggering and background reduction. I will also continue with service work, which is required for continuing authorship in the CMS Collaboration, this will involve working on the distributed computing model used by the LHC experiments, or participating in studies for the calorimetry upgrade of CMS.

\end{fmffile}
\end{document}
